
\subsection{Labor Division}

The team performed the following tasks
\bit
\item Implementation of Kcore [Jin Hu]
\item Unit Tests and visualization of results [Ye Zhou]
\item General debugging and testing[Ye Zhou and Jin Hu]
\eit


\subsection{Acknowledgement}

Thanks to Professor Christos Faloutsos for the wonderful lectures on various aspects of multimedia database and data mining topics. Thanks to TA Neil Shah for the careful design of the project and helpful feedback after each phase was completed. Thanks to Nijith Jacob and Sharif Doghmi, who took this class in a previous year for the {\em graphMiner} framework.


\subsection{Code for K-core}

\begin{lstlisting}

import argparse

from gm_params import *
from gm_sql import *
from math import sqrt
import os
import time

db_conn = None;


# Convert directed to undirected + remove multiple edges
def gm_to_undirected(rm_multiple = True):
    cur = db_conn.cursor()
    gm_sql_table_drop_create(db_conn, GM_TABLE_UNDIRECT, "src_id integer, dst_id integer, weight real")

    if rm_multiple:
        stmt = "INSERT INTO %s " % (GM_TABLE_UNDIRECT) + \
                    " SELECT src_id, dst_id, AVG(weight) FROM " + \
                    " (SELECT src_id, dst_id, weight FROM %s " % (GM_TABLE) + \
                    " UNION ALL" + \
                    " SELECT dst_id \"src_id\", src_id \"dst_id\", weight FROM %s) \"TAB\"" % (GM_TABLE) + \
                    " GROUP BY src_id, dst_id"
    else:
        stmt = "INSERT INTO %s " % (GM_TABLE_UNDIRECT) + \
                    " (SELECT src_id, dst_id, weight FROM %s " % (GM_TABLE) + \
                    " UNION ALL" + \
                    " SELECT dst_id \"src_id\", src_id \"dst_id\", weight FROM %s) " % (GM_TABLE)


    cur.execute(stmt)
    db_conn.commit()

    cur.close()

def gm_create_node_table ():
    cur = db_conn.cursor()

    gm_sql_table_drop_create(db_conn, GM_NODES, "node_id integer")

    cur.execute ("INSERT INTO %s" % GM_NODES +
                             " SELECT DISTINCT(src_id) FROM %s" % GM_TABLE_UNDIRECT)

    db_conn.commit()

    cur.close()

def gm_save_tables (dest_dir, belief):
    print "Saving tables..."

    gm_sql_save_table_to_file(db_conn, GM_DEGREE_DISTRIBUTION, "degree, count", \
                                  os.path.join(dest_dir,"degreedist.csv"), ",");
    gm_sql_save_table_to_file(db_conn, GM_INDEGREE_DISTRIBUTION, "degree, count", \
                                  os.path.join(dest_dir,"indegreedist.csv"), ",");
    gm_sql_save_table_to_file(db_conn, GM_OUTDEGREE_DISTRIBUTION, "degree, count", \
                                  os.path.join(dest_dir,"outdegreedist.csv"), ",");

    gm_sql_save_table_to_file(db_conn, GM_NODE_DEGREES, "node_id, in_degree, out_degree", \
                                  os.path.join(dest_dir,"degree.csv"), ",");

    gm_sql_save_table_to_file(db_conn, GM_PAGERANK, "node_id, page_rank", \
                                  os.path.join(dest_dir,"pagerank.csv"), ",");


    gm_sql_save_table_to_file(db_conn, GM_CON_COMP, "node_id, component_id", \
                                  os.path.join(dest_dir,"conncomp.csv"), ",");

    gm_sql_save_table_to_file(db_conn, GM_RADIUS, "node_id, radius", \
                                  os.path.join(dest_dir,"radius.csv"), ",");

    if (belief):
         gm_sql_save_table_to_file(db_conn, GM_BELIEF, "node_id, belief", \
                                  os.path.join(dest_dir,"belief.csv"), ",");

    gm_sql_save_table_to_file(db_conn, GM_EIG_VALUES, "id, value", \
                                  os.path.join(dest_dir,"eigval.csv"), ",");

    gm_sql_save_table_to_file(db_conn, GM_EIG_VECTORS, "row_id, col_id, value", \
                                  os.path.join(dest_dir,"eigvec.csv"), ",");


def kcore (args):
  global db_conn
  global GM_TABLE
  #Default Table names
  TEMP_GM_TABLE = "TEMP_GM_TABLE"
  TEMP_GM_TABLE_UNDIRECT = "TEMP_GM_TABLE_UNDIRECTED"
  TEMP_GM_NODES = "TEMP_GM_NODES"
  TEMP_GM_NODE_DEGREES = "TEMP_GM_NODE_DEGREES"
  REMOVED_NODE_TABLE = "REMOVED_NODE_TABLE"
  TEMP_GM_CON_COMP = "TEMP_GM_CON_COMP"
  temp_table = "GM_CC_TEMP"
  TEMP_GM_CON_COMP_2 = "TEMP_GM_CON_COMP_2"
  cur = db_conn.cursor()
  k = 5
  i = 2
  # create REMOVED_NODE_TABLE to record deleted node_id
  gm_sql_table_drop_create(db_conn, REMOVED_NODE_TABLE, "node_id integer")
  # create TEMP_GM_NODE_DEGREES
  gm_sql_table_drop_create(db_conn, TEMP_GM_NODE_DEGREES, "node_id integer, \
                           in_degree integer, out_degree integer")
  gm_sql_table_drop_create(db_conn, TEMP_GM_TABLE, "src_id integer, dst_id integer, weight real default 1")
  cur.execute ("INSERT INTO %s" % TEMP_GM_TABLE + " SELECT src_id, dst_id, weight FROM %s" % GM_TABLE_UNDIRECT)
  cur.execute ("INSERT INTO %s" % TEMP_GM_NODE_DEGREES + " SELECT node_id, in_degree, out_degree FROM %s" % GM_NODE_DEGREES)
  db_conn.commit()
  print "begin kcore iteration..."
  while(i<=5):
    cur.execute ("INSERT INTO %s" % REMOVED_NODE_TABLE +
                  " SELECT node_id FROM %s" % TEMP_GM_NODE_DEGREES + " WHERE in_degree < %s" % i)
    # remove degree < i nodes and associated edges in GM_TABLE and GM_TABLE_UNDIRECT
    gm_sql_table_drop_create(db_conn, TEMP_GM_TABLE, "src_id integer, dst_id integer, weight real default 1")
    gm_sql_table_drop_create(db_conn, TEMP_GM_TABLE_UNDIRECT, "src_id integer, dst_id integer, weight real default 1")
    gm_sql_table_drop_create(db_conn, TEMP_GM_NODES, "node_id integer")
    cur.execute ("INSERT INTO %s" % TEMP_GM_NODES +
                             " SELECT DISTINCT(src_id) FROM %s" % GM_TABLE_UNDIRECT + " WHERE src_id NOT IN (SELECT node_id FROM %s" % REMOVED_NODE_TABLE +" WHERE node_id IS NOT NULL) AND dst_id NOT IN (SELECT node_id FROM %s" % REMOVED_NODE_TABLE +" WHERE node_id IS NOT NULL)")
    cur.execute ("INSERT INTO %s" % TEMP_GM_TABLE + " SELECT src_id, dst_id, weight FROM %s" % GM_TABLE + " WHERE src_id NOT IN (SELECT node_id FROM %s" % REMOVED_NODE_TABLE + " WHERE node_id IS NOT NULL) AND dst_id NOT IN (SELECT node_id FROM %s" % REMOVED_NODE_TABLE +" WHERE node_id IS NOT NULL)")
    cur.execute ("INSERT INTO %s" % TEMP_GM_TABLE_UNDIRECT + " SELECT src_id, dst_id, weight FROM %s" % GM_TABLE_UNDIRECT + " WHERE src_id NOT IN (SELECT node_id FROM %s" % REMOVED_NODE_TABLE +" WHERE node_id IS NOT NULL) AND dst_id NOT IN (SELECT node_id FROM %s" % REMOVED_NODE_TABLE +" WHERE node_id IS NOT NULL)")
    db_conn.commit()

    # Recomputing node degrees..."
    gm_sql_table_drop_create(db_conn, TEMP_GM_NODE_DEGREES, "node_id integer, \
                             in_degree integer, out_degree integer")
    cur.execute ("INSERT INTO %s" % TEMP_GM_NODE_DEGREES +
                             " SELECT node_id, SUM(in_degree) \"in_degree\", SUM(out_degree) \"out_degree\" FROM " +
                             " (SELECT dst_id \"node_id\", count(*) \"in_degree\", \
                               0 \"out_degree\" FROM %s" % TEMP_GM_TABLE +
                             " GROUP BY dst_id" +
                             " UNION ALL" +
                             " SELECT src_id \"node_id\", 0 \"in_degree\", \
                               count(*) \"out_degree\" FROM %s" % TEMP_GM_TABLE +
                             " GROUP BY src_id) \"TAB\" " +
                             " GROUP BY node_id")
    db_conn.commit()
    # print "iteration when i=", i
    # gm_sql_print_table(db_conn, REMOVED_NODE_TABLE)
    # print "TEMP_GM_NODES"
    # gm_sql_print_table(db_conn, TEMP_GM_NODES)
    # print "TEMP_GM_TABLE"
    # gm_sql_print_table(db_conn, TEMP_GM_TABLE)
    # print "TEMP_GM_NODE_DEGREES"
    # gm_sql_print_table(db_conn, TEMP_GM_NODE_DEGREES)
    i = i+1

  # gm_sql_print_table(db_conn, TEMP_GM_NODES)
  # print "finish kcore iteration, now calculate connected components"
  # gm_sql_print_table(db_conn, TEMP_GM_NODE_DEGREES)

  # Connected components
  # Create CC table and initialize component id to node id

  gm_sql_create_and_insert(db_conn, TEMP_GM_CON_COMP, GM_NODES, \
                           "node_id integer, component_id integer", \
                           "node_id, component_id", "node_id, node_id")
  # gm_sql_print_table(db_conn, TEMP_GM_TABLE_UNDIRECT)
  while True:
      gm_sql_table_drop_create(db_conn, temp_table,"node_id integer, component_id integer")
      # Set component id as the min{component ids of neighbours, node's componet id}
      cur.execute("INSERT INTO %s " % temp_table +
                          " SELECT node_id, MIN(component_id) \"component_id\" FROM (" +
                              " SELECT src_id \"node_id\", MIN(component_id) \"component_id\" FROM %s, %s" % (TEMP_GM_TABLE_UNDIRECT,TEMP_GM_CON_COMP) +
                              " WHERE dst_id = node_id GROUP BY src_id" +
                              " UNION" +
                              " SELECT * FROM %s" %  TEMP_GM_CON_COMP +
                          " ) \"T\" GROUP BY node_id")
      db_conn.commit()
      diff = gm_sql_vect_diff(db_conn, TEMP_GM_CON_COMP, temp_table, "node_id", "node_id", "component_id", "component_id")
      # Copy the new component ids to the component id table
      gm_sql_create_and_insert(db_conn, TEMP_GM_CON_COMP, temp_table, \
                                  "node_id integer, component_id integer", \
                                  "node_id, component_id", "node_id, component_id")
      print "Error = " + str(diff)
      # Check whether the component ids has converged
      if (diff == 0):
          print "Component IDs has converged"
          break
  cur.execute ("SELECT count(distinct component_id) FROM %s" % TEMP_GM_CON_COMP)
  num_components = cur.fetchone()[0]
  print "Number of Components =", num_components
  print "Now output decomposition (node_id,component_id) pairs"
  cur.execute ("SELECT node_id, component_id FROM %s" % TEMP_GM_CON_COMP + " WHERE node_id IN (SELECT node_id FROM %s" % TEMP_GM_NODES + ") ORDER BY component_id")
  for x in cur:
      print x
  print "finished kcore, writing to files"
  gm_sql_table_drop_create(db_conn, TEMP_GM_CON_COMP_2,"node_id integer, component_id integer")
  cur.execute ("INSERT INTO %s" % TEMP_GM_CON_COMP_2 + " SELECT node_id, component_id FROM %s" % TEMP_GM_CON_COMP + " WHERE node_id IN (SELECT node_id FROM %s" % TEMP_GM_NODES + ") ORDER BY component_id")
  gm_sql_save_table_to_file(db_conn, TEMP_GM_CON_COMP_2, "node_id, component_id", \
                                os.path.join(args.dest_dir,"kcore_conncomp.csv"), ",");
  # Drop temp tables
  gm_sql_table_drop(db_conn, temp_table)
  gm_sql_table_drop(db_conn, TEMP_GM_CON_COMP)
  cur.close()
  return

#Project Tasks

#Task 1: Degree distribution
#-----------------------------------------------------------------------------#
def gm_node_degrees ():
    cur = db_conn.cursor()

    # Create Table to store node degrees
    # If the graph is undirected, all the degree values will be the same
    print "Computing Node degrees..."

    gm_sql_table_drop_create(db_conn, GM_NODE_DEGREES, "node_id integer, \
                             in_degree integer, out_degree integer")

    cur.execute ("INSERT INTO %s" % GM_NODE_DEGREES +
                             " SELECT node_id, SUM(in_degree) \"in_degree\", SUM(out_degree) \"out_degree\" FROM " +
                             " (SELECT dst_id \"node_id\", count(*) \"in_degree\", \
                               0 \"out_degree\" FROM %s" % GM_TABLE +
                             " GROUP BY dst_id" +
                             " UNION ALL" +
                             " SELECT src_id \"node_id\", 0 \"in_degree\", \
                               count(*) \"out_degree\" FROM %s" % GM_TABLE +
                             " GROUP BY src_id) \"TAB\" " +
                             " GROUP BY node_id")
    db_conn.commit()

    cur.close()


# Degree distribution
def gm_degree_distribution (undirected):

    cur = db_conn.cursor()
    print "Computing Degree distribution of the nodes..."

    gm_sql_table_drop_create(db_conn, GM_DEGREE_DISTRIBUTION, "degree integer, count integer")
    gm_sql_table_drop_create(db_conn, GM_INDEGREE_DISTRIBUTION, "degree integer, count integer")
    gm_sql_table_drop_create(db_conn, GM_OUTDEGREE_DISTRIBUTION, "degree integer, count integer")

    cur.execute ("INSERT INTO %s" % GM_INDEGREE_DISTRIBUTION +
                            " SELECT in_degree \"degree\", count(*) FROM %s" % (GM_NODE_DEGREES) +
                            " GROUP BY in_degree");

    cur.execute ("INSERT INTO %s" % GM_OUTDEGREE_DISTRIBUTION +
                            " SELECT out_degree \"degree\", count(*) FROM %s" % (GM_NODE_DEGREES) +
                            " GROUP BY out_degree");

    if (undirected):
        # Degree distribution is same as in/out degree distribution for undirected graphs
        cur.execute ("INSERT INTO %s" % GM_DEGREE_DISTRIBUTION +
                            " SELECT * FROM %s" % GM_INDEGREE_DISTRIBUTION);
    else:
        cur.execute ("INSERT INTO %s" % GM_DEGREE_DISTRIBUTION +
                            " SELECT in_degree+out_degree \"degree\", count(*) FROM %s" % (GM_NODE_DEGREES) +
                            " GROUP BY in_degree+out_degree");

    db_conn.commit()
    cur.close()

# Task 2: PageRank
# ------------------------------------------------------------------------- #
def gm_pagerank (num_nodes, max_iterations = gm_param_pr_max_iter, \
                    stop_threshold = gm_param_pr_thres, damping_factor = gm_param_pr_damping):

    offset_table = "GM_PR_OFFSET"
    next_table = "GM_PR_NEXT"
    norm_table = "GM_PR_NORM"

    cur = db_conn.cursor();
    print "Computing PageRanks..."

    gm_sql_table_drop_create(db_conn, norm_table,"src_id integer, dst_id integer, weight double precision")

    # Create normalized weighted table
    cur.execute("INSERT INTO %s " % norm_table +
            " SELECT src_id, dst_id, weight/weight_sm \"weight\" FROM %s \"TAB1\", " % (GM_TABLE) +
            " (SELECT src_id \"node_id\", sum(weight) \"weight_sm\" FROM %s GROUP BY src_id) \"TAB2\" " % (GM_TABLE) +
            " WHERE \"TAB1\".src_id = \"TAB2\".node_id")
    db_conn.commit();

    # Create PageRank Table and initialize to 1/n
    gm_sql_create_and_insert(db_conn, GM_PAGERANK, GM_NODES, \
                             "node_id integer, page_rank double precision default %s" % (1.0/num_nodes), \
                             "node_id", "node_id")

    # Create offset table and initialize to 1-c/n
    gm_sql_create_and_insert(db_conn, offset_table, GM_NODES, \
                             "node_id integer, page_rank double precision default %s" % ((1.0-damping_factor)/num_nodes), \
                             "node_id", "node_id")
    num_iterations = 0
    while True:
        # Create Table to store the next pageRank
        gm_sql_table_drop_create(db_conn, next_table,"node_id integer, page_rank double precision")

        # Compute Next PageRank
        cur.execute ("INSERT INTO %s " % next_table +
                                " SELECT node_id, SUM(page_rank) FROM (" +
                                " SELECT dst_id \"node_id\", SUM(%s*weight*page_rank) \"page_rank\" " % damping_factor +
                                " FROM %s, %s" % (norm_table, GM_PAGERANK) +
                                " WHERE src_id = node_id GROUP BY dst_id" +
                                " UNION ALL" +
                                " SELECT node_id, page_rank * val \"page_rank\" " +
                                " FROM %s, (SELECT SUM(page_rank) \"val\" FROM %s) \"PRSUM\" " % (offset_table, GM_PAGERANK) +
                                " ) \"TAB\" GROUP BY node_id" )

        db_conn.commit()

        diff = gm_sql_vect_diff(db_conn, GM_PAGERANK, next_table, \
                                "node_id", "node_id", "page_rank", "page_rank")

        # Copy the new page rank to the page rank table
        gm_sql_create_and_insert(db_conn, GM_PAGERANK, next_table, \
                                    "node_id integer, page_rank double precision", \
                                    "node_id, page_rank", "node_id, page_rank")

        num_iterations = num_iterations + 1
        print "Iteration = %d, Error = %f" % (num_iterations, diff)

        if (diff<=stop_threshold or num_iterations>=max_iterations):
            break

    # Drop temp tables
    gm_sql_table_drop(db_conn, offset_table)
    gm_sql_table_drop(db_conn, next_table)
    gm_sql_table_drop(db_conn, norm_table)

    cur.close()

# Task 3: Weakly Connected Components
#-------------------------------------------------------------#
def gm_connected_components (num_nodes):
    temp_table = "GM_CC_TEMP"
    cur = db_conn.cursor()
    print 'Computing Weakly Connected Components...'

    # Create CC table and initialize component id to node id
    gm_sql_create_and_insert(db_conn, GM_CON_COMP, GM_NODES, \
                             "node_id integer, component_id integer", \
                             "node_id, component_id", "node_id, node_id")

    while True:
        gm_sql_table_drop_create(db_conn, temp_table,"node_id integer, component_id integer")

        # Set component id as the min{component ids of neighbours, node's componet id}
        cur.execute("INSERT INTO %s " % temp_table +
                            " SELECT node_id, MIN(component_id) \"component_id\" FROM (" +
                                " SELECT src_id \"node_id\", MIN(component_id) \"component_id\" FROM %s, %s" % (GM_TABLE_UNDIRECT,GM_CON_COMP) +
                                " WHERE dst_id = node_id GROUP BY src_id" +
                                " UNION" +
                                " SELECT * FROM %s" %  GM_CON_COMP +
                            " ) \"T\" GROUP BY node_id")
        db_conn.commit()

        diff = gm_sql_vect_diff(db_conn, GM_CON_COMP, temp_table, "node_id", "node_id", "component_id", "component_id")

        # Copy the new component ids to the component id table
        gm_sql_create_and_insert(db_conn, GM_CON_COMP, temp_table, \
                                    "node_id integer, component_id integer", \
                                    "node_id, component_id", "node_id, component_id")

        print "Error = " + str(diff)
        # Check whether the component ids has converged
        if (diff == 0):
            print "Component IDs has converged"
            break

    cur.execute ("SELECT count(distinct component_id) FROM %s" % GM_CON_COMP)
    num_components = cur.fetchone()[0]

    print "Number of Components =", num_components
    cur.close()

    # Drop temp tables
    gm_sql_table_drop(db_conn, temp_table)

# Task 4: Radius of every node
#-------------------------------------------------------------#
def gm_all_radius (num_nodes, max_iter = gm_param_radius_max_iter):

    hop_table = "GM_RD_HOP"
    max_hop_ngh = "GM_RD_MAX_HOP_NGH"

    cur = db_conn.cursor()
    print 'Computing radius of every node...'

    # initialize hop 0 table's hash
    gm_sql_create_and_insert(db_conn, hop_table+"0", GM_NODES, \
                             "node_id integer, hash integer", \
                             "node_id, hash", "node_id, (((node_id%%%s+1)#(node_id%%%s))+1)/2" % (num_nodes, num_nodes))

    for cur_hop in range(1,max_iter+1):
        print "Hop number : " + str(cur_hop)

        # create ith hop table
        cur_hop_table = hop_table+str(cur_hop)
        prev_hop_table = hop_table+str(cur_hop-1)
        gm_sql_table_drop_create(db_conn, cur_hop_table,"node_id integer, hash integer")
        cur.execute("INSERT INTO %s " % cur_hop_table +
                            " SELECT node_id, bit_or(hash) FROM ( " +
                            " SELECT src_id \"node_id\", bit_or(hash) \"hash\" " +
                            " FROM %s,%s" % (GM_TABLE_UNDIRECT, prev_hop_table) +
                            " WHERE dst_id = node_id GROUP BY src_id " +
                            " UNION ALL" +
                            " SELECT * FROM %s ) \"TAB\" GROUP BY node_id" % (prev_hop_table))
        db_conn.commit()

        # Check convergence
        diff = gm_sql_vect_diff(db_conn, cur_hop_table, prev_hop_table, "node_id", "node_id", "hash", "hash")

        print "Current Error = " + str(diff)
        if (diff==0):
            print "Convergence acheived"
            break

    nghbourhd_func = "2^(floor(log(2,hash)+1))/0.77351"
    gm_sql_create_and_insert(db_conn, max_hop_ngh, cur_hop_table, \
                             "id integer, value double precision", \
                             "id, value", "node_id, %s" % (nghbourhd_func))



    gm_sql_table_drop_create(db_conn, GM_RADIUS,"node_id integer, radius integer")

    for i in range(0,cur_hop+1):
        print "Getting nodes with eff. radius " + str(i)
        # effective radius is the hop at which neighbour fucntion value exceeds
        # 0.9 * the value at max hop
        cur.execute("INSERT INTO %s" % GM_RADIUS +
                        " SELECT node_id, %s \"radius\" FROM %s, %s " % (i, hop_table+str(i), max_hop_ngh) +
                        " WHERE node_id = id AND %s>=0.9*value " % (nghbourhd_func))

        db_conn.commit()
        cur.execute("DELETE FROM %s WHERE id IN(SELECT node_id FROM %s)" % (max_hop_ngh, GM_RADIUS));
        db_conn.commit()


    cur.execute ("SELECT max(radius) FROM %s" % GM_RADIUS)
    max_radius = cur.fetchone()[0]
    print "Maximum effective radius =", max_radius

    # drop temp tables
    gm_sql_table_drop(db_conn, max_hop_ngh)
    for i in range(0,cur_hop+1):
        gm_sql_table_drop(db_conn, hop_table+str(i))


    cur.close()

# Task 5: Eigen values
# ------------------------------------------------------------------------- #
# The adjacency matrix should be symmetric

def gm_eigen_QR_decompose(T, n, Q, R):
    G = "GM_QR_DECOMPOSE_GIVENS"
    temp_table = "GM_QR_DECOMPOSE_TEMP"
    I = "GM_QR_DECOMPOSE_IDENTITY"

    cur = db_conn.cursor();

    gm_sql_table_drop_create(db_conn, R,"row_id integer, col_id integer, value double precision")

    # Initialize R = T
    cur.execute("INSERT INTO %s" % (R) + " SELECT * FROM %s" % (T))
    db_conn.commit()


    for i in range(1,n):
        # Compute the givens matrix
        cur.execute("SELECT value FROM %s " % (R) +
                        "WHERE col_id = %s AND row_id >= %s ORDER BY row_id" % (str(i), str(i)) )

        c = cur.fetchone()[0]
        s = cur.fetchone()[0]
        r = sqrt(c*c + s*s)
        c = c/r
        s = -s/r

        gm_sql_table_drop_create(db_conn, G,"row_id integer, col_id integer, value double precision")
        cur.execute("INSERT INTO %s" % (G) + " SELECT * FROM %s" % (I))
        cur.execute('UPDATE %s' % (G) + ' SET value = %s WHERE row_id = %s AND col_id = %s' %(str(c),str(i),str(i)))
        cur.execute('UPDATE %s' % (G) + ' SET value = %s WHERE row_id = %s AND col_id = %s' %(str(c),str(i+1),str(i+1)))
        cur.execute('INSERT INTO %s' % (G) + ' VALUES (%s,%s,%s)' %(str(i),str(i+1),str(-s)))
        cur.execute('INSERT INTO %s' % (G) + ' VALUES (%s,%s,%s)' %(str(i+1),str(i),str(s)))
        db_conn.commit()

        # Compute Q
        if i == 1:
            # insert G*
            gm_sql_table_drop_create(db_conn, Q,"row_id integer, col_id integer, value double precision")
            cur.execute("INSERT INTO %s" % (Q) + " SELECT \"col_id\" row_id, \"row_id\" col_id, value FROM %s" % (G))
        else:
            gm_sql_table_drop_create(db_conn, temp_table,"row_id integer, col_id integer, value double precision")
            gm_sql_mat_mat_multiply (db_conn, Q, G, temp_table, "col_id", "col_id", "value", "value",
                                             "value", "row_id", "row_id", "row_id", "row_id")
            gm_sql_table_drop_create(db_conn, Q,"row_id integer, col_id integer, value double precision")
            cur.execute("INSERT INTO %s" % (Q) + " SELECT * FROM %s" % (temp_table))

        db_conn.commit()
        # Compute R
        gm_sql_table_drop_create(db_conn, temp_table,"row_id integer, col_id integer, value double precision")
        gm_sql_mat_mat_multiply (db_conn, G, R, temp_table, "col_id", "row_id", "value", "value",
                                             "value", "row_id", "col_id", "row_id", "col_id")
        gm_sql_table_drop_create(db_conn, R,"row_id integer, col_id integer, value double precision")
        cur.execute("INSERT INTO %s" % (R) + " SELECT * FROM %s" % (temp_table))

        db_conn.commit()

    cur.close()
    # Drop temp tables
    gm_sql_table_drop(db_conn, G)

    gm_sql_table_drop(db_conn, temp_table)



def gm_eigen_QR_iterate(T, n, EVal, EVec, steps, err):

    Q = "GM_QR_Q"
    R = "GM_QR_R"
    temp_table = "GM_QR_TEMP"
    I = "GM_QR_DECOMPOSE_IDENTITY"
    print 'Performing QR Algorithm. Max Iters=%s, Stop threshold=%s' % (steps, err)

    cur = db_conn.cursor();

    gm_sql_table_drop_create(db_conn, EVal,"row_id integer, col_id integer, value double precision")
    gm_sql_table_drop_create(db_conn, EVec,"row_id integer, col_id integer, value double precision")

    gm_sql_table_drop_create(db_conn, I,"row_id integer, col_id integer, value double precision")
    gm_sql_load_table(db_conn, I, [str(i) + " " + str(i) + " " + str(1) for i in range(1,n+1)])

    cur.execute("INSERT INTO %s" % (EVal) + " SELECT * FROM %s" % (T))
    db_conn.commit()

    for i in range(1,steps+1):

        try:
            gm_eigen_QR_decompose(EVal, n, Q, R)
        except psycopg2.DataError:
            db_conn.commit()
            break

        gm_sql_table_drop_create(db_conn, EVal,"row_id integer, col_id integer, value double precision")

        # Set EVal as RQ
        gm_sql_mat_mat_multiply (db_conn, R, Q, EVal, "col_id", "row_id", "value", "value",
                                             "value", "row_id", "col_id", "row_id", "col_id")

        if i==1:
            # Copy Q to EVec
            cur.execute("INSERT INTO %s" % (EVec) + " SELECT * FROM %s" % (Q))
            db_conn.commit()
        else:
            # Set EVec = EVec * Q
            gm_sql_table_drop_create(db_conn, temp_table,"row_id integer, col_id integer, value double precision")
            gm_sql_mat_mat_multiply (db_conn, EVec, Q, temp_table, "col_id", "row_id", "value", "value",
                                             "value", "row_id", "col_id", "row_id", "col_id")

            gm_sql_table_drop_create(db_conn, EVec,"row_id integer, col_id integer, value double precision")
            cur.execute("INSERT INTO %s" % (EVec) + " SELECT * FROM %s" % (temp_table))
            db_conn.commit()

            cur.execute("SELECT max(abs(value)) FROM %s" % (EVal) + " WHERE row_id <> col_id" )
            cur_err = cur.fetchone()[0]

            print "QR Algorithm Error = %s" % cur_err
            if cur_err <= err:
                break

    cur.close()

    # Drop temp tables
    gm_sql_table_drop(db_conn, Q)
    gm_sql_table_drop(db_conn, R)
    gm_sql_table_drop(db_conn, temp_table)
    gm_sql_table_drop(db_conn, I)



def gm_eigen (steps, num_nodes, err1, err2, adj_table=GM_TABLE_UNDIRECT):

    QR_max_iter = gm_param_qr_max_iter
    QR_stop_threshold = gm_param_qr_thres

    basis_vect_0 = "GM_EG_BASIS_VECT0"
    basis_vect_1 = "GM_EG_BASIS_VECT1"
    next_basis_vect = "GM_EG_BASIS_VECT_NEXT"
    temp_vect = "GM_EG_TEMP_VECT"
    temp_vect2 = "GM_EG_TEMP_VECT2"
    temp_vect3 = "GM_EG_TEMP_VECT3"
    basis = "GM_EG_BASIS"
    tridiag_table = "GM_EG_TRIDIAGONAL"
    diag_table = "GM_EG_DIAG"
    eigvec_table = "GM_EG_VEC"

    cur = db_conn.cursor();
    print "Computing Eigenvalues..."

    # create basis vectors
    gm_sql_vector_random(db_conn, basis_vect_1)
    gm_sql_create_and_insert(db_conn, basis_vect_0, GM_NODES, \
                             "id integer, value double precision", \
                             "id, value", "node_id, 0")

    # Create table to store the basis vectors
    gm_sql_table_drop_create(db_conn, basis,"row_id integer, col_id integer, value double precision")

    gm_sql_table_drop_create(db_conn, tridiag_table,"row_id integer, col_id integer, value double precision")

    beta_0 = 0
    beta_1 = 0
    alph_1 = 0

    for i in range(1, steps+1):
        print "Iteration No: " + str(i)

        # Get the next basis
        gm_sql_table_drop_create(db_conn, next_basis_vect,"id integer, value double precision")
        gm_sql_adj_vect_multiply(db_conn, adj_table, basis_vect_1, next_basis_vect, "dst_id",
                                    "id", "id", "value", "value", "src_id")

        alph_1 = gm_sql_vect_dotproduct (db_conn, next_basis_vect, basis_vect_1, "id", "id", "value", "value")

        gm_sql_table_drop_create(db_conn, temp_vect,"id integer, value double precision")
        # Orthogonalize with previous two basis vectors
        cur.execute("INSERT INTO %s " % (temp_vect) +
                        " (SELECT \"VECT_NEW\".id, " +
                        " (\"VECT_NEW\".value - (%s * \"VECT_0\".value) - (%s * \"VECT_1\".value)) \"value\"" %
                                                                (beta_0, alph_1) +
                        " FROM %s \"VECT_NEW\", %s \"VECT_0\", %s \"VECT_1\" " %
                                                                (next_basis_vect, basis_vect_0, basis_vect_1) +
                        " WHERE \"VECT_NEW\".id = \"VECT_0\".id AND \"VECT_0\".id = \"VECT_1\".id)")

        db_conn.commit()

        # Insert values into the tridiagonal table
        cur.execute("INSERT INTO %s" % (tridiag_table) + " VALUES(%s,%s,%s)" % (i,i,alph_1))
        if i>1:
            cur.execute("INSERT INTO %s" % (tridiag_table) + " VALUES(%s,%s,%s)" % (i-1,i, beta_0))
            cur.execute("INSERT INTO %s" % (tridiag_table) + " VALUES(%s,%s,%s)" % (i,i-1, beta_0))

        db_conn.commit()

        # Save the basis vector
        cur.execute("INSERT INTO %s " % (basis) +
                    "SELECT id \"row_id\", %s \"col_id\", value " % (i) +
                    "FROM %s" % (basis_vect_1))

        db_conn.commit()

        if i>1:
            gm_eigen_QR_iterate(tridiag_table, i, diag_table, eigvec_table, QR_max_iter, QR_stop_threshold)

            for j in range(1,i+1):
                cur.execute("SELECT abs(value) FROM %s" % (eigvec_table) +
                                " WHERE col_id=%s AND row_id=%s" % (j,i))

                thr = cur.fetchone()
                if thr:
                    thr = thr[0]
                else:
                    thr = 0

                if thr <= err1:
                    print "Performing SO with EigenVector " + str(j)
                    # Get corresponding eigenvector
                    gm_sql_table_drop_create(db_conn, temp_vect2,"id integer, value double precision")

                    gm_sql_mat_colvec_multiply (db_conn, basis, eigvec_table, temp_vect2, "col_id", "row_id",
                                    "id", "value", "value", "value", "row_id", "col_id="+str(j))

                    # Selectively orthogonalize
                    r = gm_sql_vect_dotproduct (db_conn, temp_vect2, temp_vect, "id", "id", "value", "value")

                    gm_sql_table_drop_create(db_conn, temp_vect3,"id integer, value double precision")
                    cur.execute("INSERT INTO %s " % (temp_vect3) +
                                " (SELECT \"VECT1\".id, " +
                                " (\"VECT1\".value - (%s * \"VECT2\".value)) \"value\"" % (r) +
                                " FROM %s \"VECT1\", %s \"VECT2\" " % (temp_vect, temp_vect2) +
                                " WHERE \"VECT1\".id = \"VECT2\".id)")

                    db_conn.commit()

                    gm_sql_table_drop_create(db_conn, temp_vect,"id integer, value double precision")
                    cur.execute("INSERT INTO %s" % (temp_vect) + " SELECT * FROM %s" % (temp_vect3))

                    db_conn.commit()

        beta_1 = gm_sql_normalize_vector (db_conn, temp_vect, "value");


        if abs(beta_1) <= err2:
            break

        # Prepare for next iteration
        gm_sql_table_drop_create(db_conn, basis_vect_0,"id integer, value double precision")
        cur.execute("INSERT INTO %s" % (basis_vect_0) + " SELECT * FROM %s" % (basis_vect_1))
        db_conn.commit()

        gm_sql_table_drop_create(db_conn, basis_vect_1,"id integer, value double precision")
        cur.execute("INSERT INTO %s" % (basis_vect_1) + " SELECT * FROM %s" % (temp_vect))
        db_conn.commit()

        beta_0 = beta_1

    # Get the eigen values and eigen vectors
    gm_eigen_QR_iterate(tridiag_table, i, diag_table, eigvec_table, QR_max_iter, QR_stop_threshold)

    gm_sql_table_drop_create(db_conn, GM_EIG_VALUES,"id integer, value double precision")

    print "Getting EigenValues..."
    # Get top eigen values
    cur.execute("INSERT INTO %s" % (GM_EIG_VALUES) +
                    " SELECT col_id \"id\", value \"value\" FROM %s" % (diag_table) +
                    " WHERE col_id = row_id ORDER BY abs(value) desc")

    db_conn.commit()

    # Get the top k eigenvectors
    print "Getting top %s EigenVectors..." % gm_param_eig_k
    cur2 = db_conn.cursor();
    gm_sql_table_drop_create(db_conn, GM_EIG_VECTORS,"row_id integer, col_id integer, value double precision")

    cur.execute("SELECT id FROM %s ORDER BY abs(value) desc LIMIT %s " % (GM_EIG_VALUES,gm_param_eig_k))
    for idx in cur:
        i = idx[0]
        print "Getting eigenvector %s" % i
        gm_sql_table_drop_create(db_conn, temp_vect2,"id integer, value double precision")
        gm_sql_mat_colvec_multiply (db_conn, basis, eigvec_table, temp_vect2, "col_id", "row_id",
                                "id", "value", "value", "value", "row_id", "col_id="+str(i))

        cur2.execute("INSERT INTO %s SELECT id \"row_id\", %s \"col_id\", value" % (GM_EIG_VECTORS,i) +
                                " FROM %s" % (temp_vect2))
        db_conn.commit()

    cur2.close()

#
#    gm_sql_mat_mat_multiply (db_conn, basis, eigvec_table, GM_EIG_VECTORS, "col_id", "row_id", "value", "value",
#                                             "value", "row_id", "col_id", "row_id", "col_id")

    print "EigenValues computed: "
    gm_sql_print_table(db_conn, GM_EIG_VALUES)
    #gm_sql_print_table(db_conn, GM_EIG_VECTORS)

    cur.close()
    # Drop temp tables
    gm_sql_table_drop(db_conn, basis_vect_0)
    gm_sql_table_drop(db_conn, basis_vect_1)
    gm_sql_table_drop(db_conn, next_basis_vect)
    gm_sql_table_drop(db_conn, temp_vect)
    gm_sql_table_drop(db_conn, temp_vect2)
    gm_sql_table_drop(db_conn, temp_vect3)
    gm_sql_table_drop(db_conn, basis)
    gm_sql_table_drop(db_conn, tridiag_table)
    gm_sql_table_drop(db_conn, diag_table)
    gm_sql_table_drop(db_conn, eigvec_table)


# Task 6: Fast Belief Propagation
# ------------------------------------------------------------------------- #
def gm_belief_propagation(belief_file, delim, undirected, \
                max_iterations = gm_param_bp_max_iter, stop_threshold = gm_param_bp_thres):

    next_table = "GM_BP_NEXT"
    print "Computing belief propagation..."

    # BP require that the graph is undirected.
    if (undirected):
        degree_col = "out_degree"
    else:
        degree_col = "out_degree+in_degree"

    cur = db_conn.cursor()
    cur.execute ("SELECT MAX(%s), SUM(%s), SUM((%s)*(%s))" % (degree_col, degree_col, degree_col, degree_col) +
                 "FROM %s" % GM_NODE_DEGREES)
    max_deg, sum_deg, sum_deg2  = cur.fetchone()

    c1 = 2+sum_deg
    c2 = sum_deg2 -1

    h = max(1 / (float)(2 + 2 * max_deg), sqrt((-c1 + sqrt(c1*c1 + 4*c2))/(8*c2)))
    print "Homophily factor = " + str(h)

    a = (4*h*h)/(1-4*h*h)
    c = (2*h)/(1-4*h*h)

    print "Getting the priors..."
    # Get the belief priors.
    gm_sql_table_drop_create(db_conn, GM_BELIEF_PRIOR, "node_id integer, belief double precision")
    gm_sql_load_table_from_file(db_conn, GM_BELIEF_PRIOR, "node_id, belief", belief_file, delim)

    # Initialize belief table as belief priors
    gm_sql_create_and_insert(db_conn, GM_BELIEF, GM_BELIEF_PRIOR, \
                             "node_id integer, belief double precision", \
                             "node_id, belief", "node_id, belief")

    num_iterations = 0
    while True:
        # Create Table to store the next belief
        gm_sql_table_drop_create(db_conn, next_table,"node_id integer, belief double precision")

        # Compute next belief
        cur.execute ("INSERT INTO %s " % next_table +
                                " SELECT node_id, SUM(belief) FROM (" +
                                " SELECT src_id \"node_id\", %s * SUM(belief) \"belief\" " % c +
                                " FROM %s, %s" % (GM_TABLE_UNDIRECT, GM_BELIEF) +
                                " WHERE dst_id = node_id GROUP BY src_id" +
                                " UNION ALL" +
                                " SELECT \"D\".node_id \"node_id\", %s*(%s)*belief \"belief\"" % (-a, degree_col) +
                                " FROM %s \"D\", %s \"B\" " % (GM_NODE_DEGREES, GM_BELIEF) +
                                " WHERE \"D\".node_id = \"B\".node_id" +
                                " UNION ALL" +
                                " SELECT * FROM %s " % GM_BELIEF_PRIOR +
                                " ) \"TAB\" GROUP BY node_id" )

        db_conn.commit()

        diff = gm_sql_vect_diff(db_conn, GM_BELIEF, next_table, "node_id", "node_id", "belief", "belief")

        # Recreate Belief table and copy values
        gm_sql_create_and_insert(db_conn, GM_BELIEF, next_table, \
                                 "node_id integer, belief double precision", \
                                 "node_id, belief", "node_id, belief")

        num_iterations = num_iterations + 1
        print "Iteration = %d, Error = %f" % (num_iterations, diff)

        if (diff<=stop_threshold or num_iterations>max_iterations):
            break


    # Drop temp tables
    gm_sql_table_drop(db_conn, next_table)

    cur.close()



# Task 7: Triangle counting
# ------------------------------------------------------------------------- #
def gm_naive_triangle_count(adj_table=GM_TABLE_UNDIRECT):

    temp_table = "GM_TRIANG_TEMP"
    temp_table2 = "GM_TRIANG_TEMP2"
    temp_table3 = "GM_TRIANG_TEMP3"

    cur = db_conn.cursor()
    gm_sql_table_drop_create(db_conn, temp_table,"row_id integer, col_id integer, value double precision")
    gm_sql_table_drop_create(db_conn, temp_table2,"row_id integer, col_id integer, value double precision")
    gm_sql_table_drop_create(db_conn, temp_table3,"row_id integer, col_id integer, value double precision")

    # Copy the adjacency matrix
    cur.execute("INSERT INTO %s" % (temp_table) + \
                " SELECT src_id \"row_id\", dst_id \"col_id\", 1 \"value\" FROM %s" % (adj_table))

    db_conn.commit()

    # Compute A^2
    gm_sql_mat_mat_multiply (db_conn, temp_table, temp_table, temp_table2, "col_id", "row_id", "value", "value",
                                             "value", "row_id", "col_id", "row_id", "col_id")
    # Compute A^3
    gm_sql_mat_mat_multiply (db_conn, temp_table2, temp_table, temp_table3, "col_id", "row_id", "value", "value",
                                             "value", "row_id", "col_id", "row_id", "col_id")


    cnt = gm_sql_mat_trace(db_conn, temp_table3, "row_id", "col_id", "value")

    print "Number of Triangles(naive) = " + (str(cnt/6))

    cur.close()

    # Drop temp tables
    gm_sql_table_drop(db_conn, temp_table)
    gm_sql_table_drop(db_conn, temp_table2)
    gm_sql_table_drop(db_conn, temp_table3)


def gm_eigen_triangle_count():

    cur = db_conn.cursor()
    #gm_eigen(steps, num_nodes, err1, err2, adj_table)
    print "Computing the count of triangles..."

    cur.execute("SELECT sum(value^3) FROM %s" % (GM_EIG_VALUES))
    cnt = cur.fetchone()[0]

    print "Number of Triangles = " + (str(cnt/6))

    cur.close()


# Innovative Task : Anomaly Detection for unidrected graphs
def gm_anomaly_detection():
    cur = db_conn.cursor()
    gm_sql_table_drop_create(db_conn, GM_EGONET,"node_id integer, edge_cnt integer, wgt_sum double precision")

    print "Extracting Features from Egonets"

    start_time = time.time()
    cur.execute("INSERT INTO %s " % (GM_EGONET) +
                        " SELECT node_id, sum(edge_cnt) \"edge_cnt\", sum(wgt_sum) \"wgt_sum\" FROM" +
                        " (SELECT \"T2\".dst_id \"node_id\", count(*)/2 \"edge_cnt\", sum(\"T2\".weight)/2 \"wgt_sum\" " +
                        " FROM %s \"T1\", %s \"T2\", %s \"T3\" " % (GM_TABLE_UNDIRECT, GM_TABLE_UNDIRECT, GM_TABLE_UNDIRECT) +
                        " WHERE \"T1\".src_id = \"T2\".src_id AND \"T1\".dst_id = \"T3\".dst_id AND \"T2\".dst_id=\"T3\".src_id" +
                        " GROUP BY \"T2\".dst_id" +
                        " UNION ALL" +
                        " SELECT src_id \"node_id\", count(*) \"edge_cnt\", sum(weight) \"wgt_sum\" " +
                        " FROM %s GROUP BY src_id) \"TAB\" " % (GM_TABLE_UNDIRECT) +
                        " GROUP BY node_id");

    db_conn.commit();
    print "Time taken = " + str(time.time()-start_time)



def main():
    global db_conn
    global GM_TABLE
     # Command Line processing
    parser = argparse.ArgumentParser(description="Graph Miner Using SQL v1.0")
    parser.add_argument ('--file', dest='input_file', type=str, required=True,
                         help='Full path to the file to load from. For weighted \
                         graphs, the file should have the format (<src_id>, <dst_id>, <weight>) \
                         . If unweighted please run with --unweighted option. To specify a \
                         delimiter other than "," (default), use --delim option. \
                         NOTE: The file should have proper permissions set for \
                         the postgres user.'
                         )
    parser.add_argument ('--delim', dest='delimiter', type=str, default=',',
                         help='Delimiter that separate the columns in the input file. default ","')

    parser.add_argument ('--unweighted', dest='unweighted', action='store_const', const=True, default=False,
                         help='For unweighted graphs. The input file should be of the form \
                         (<src_id>, <dst_id>). For algorithms that require weighted graphs, default weight \
                         of 1 will be used')

    parser.add_argument ('--undirected', dest='undirected', action='store_const', const=True, default=False,
                         help='Treat the graph as undirected instead of directed (default). If this is set \
                         the input graph is first converted into an undirected version by adding reversed edges \
                         with same weight. NOTE: Graph algorithms like eigen values, triangle counting, \
                         connected components etc require undirected graphs and such algorithms work with \
                         undirected version of the graph irrespective of whether this option is set.')

    parser.add_argument ('--dest_dir', dest='dest_dir', type=str, required=True,
                         help='Full path to the directory where the output tables are saved')

    parser.add_argument ('--belief_file', dest='belief_file', type=str, default='',
                         help='Full path to belief priors file. The file should be in the format \
                         (<node_id>, <belief>). Specify a different delimiter with --delim option.\
                         The prior beliefs are expected to be centered around 0. i.e. positive \
                         nodes have priors >0, negative nodes <0 and unknown nodes 0. ')

    args = parser.parse_args()

    try:
        # Run the various graph algorithm below
        db_conn = gm_db_initialize()

        gm_sql_table_drop_create(db_conn, GM_TABLE, "src_id integer, dst_id integer, weight real default 1")
        if (args.unweighted):
            col_fmt = "src_id, dst_id"
        else:
            col_fmt = "src_id, dst_id, weight"

        gm_sql_load_table_from_file(db_conn, GM_TABLE, col_fmt, args.input_file, args.delimiter)

        gm_to_undirected(False)

        if (args.undirected):
            GM_TABLE = GM_TABLE_UNDIRECT

        # Create table of node ids
        gm_create_node_table()
        # Get number of nodes
        cur = db_conn.cursor()
        cur.execute("SELECT count(*) from %s" % GM_NODES)
        num_nodes = cur.fetchone()[0]

        gm_node_degrees()

        # Tasks
        gm_degree_distribution(args.undirected)                 # Degree distribution
        kcore(args)
        gm_pagerank(num_nodes)                                  # Pagerank
        gm_connected_components(num_nodes)                      # Connected components
        gm_eigen(gm_param_eig_max_iter, num_nodes, gm_param_eig_thres1, gm_param_eig_thres2)
        gm_all_radius(num_nodes)
        if (args.belief_file):
            gm_belief_propagation(args.belief_file, args.delimiter, args.undirected)


        gm_eigen_triangle_count()
        #gm_naive_triangle_count()

        # Save tables to disk
        gm_save_tables(args.dest_dir, args.belief_file)
        #gm_anomaly_detection()

        gm_db_bubye(db_conn)
    except:
        print "Unexpected error:", sys.exc_info()[0]
        if (db_conn):
            gm_db_bubye(db_conn)

        raise

if __name__ == '__main__':
    main()


\end{lstlisting}